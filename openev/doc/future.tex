\documentclass{article}

\begin{document}

\title{Improvements to OpenEV}
\date{2000-06-20}
\author{OpenEV Project}
\maketitle

\section{Future Directions}

As OpenEV matures, new tools and layer classes, and common UI elements
will be added to enhance its capabilities.  The object oriented
structure ensures that this growth is possible.

The follow sections list ideas that could be added to OpenEV.

\subsection{3D}

One particular area for improvement in OpenEV is in its handling of 3D imagery.

\begin{itemize}
\item continuous LOD approach to rendering the 3D mesh is required to allow for
interactive frame rates when dealing with large elevation datasets.
\item intellegent tesselation of mesh
\item 3D datasets, such as scene graph elements (e.g. city buildings) could be added.
\item LUT for mesh would allow elevation data to be visualized without a drape
\item lighting could be added for more realistic view, this would also facilitate viewing
elevation data without a drape.
\end{itemize}


\subsection{R\&D Tools}

From its conception OpenEV has been designed to be a R\&D tool.  

\begin{itemize}
\item Another important direction is coupling the display capabilities with
a Python shell window, to create an interactive numerical analysis and
image manipulation environment in the vein of Research Systems's IDL,
or The Mathwork's MATLAB.  This will eventually become an integrated
environment where the user is able to combine plug-in processing and
analysis functions, plotting capabilities, image display, point and
vector marking, and a powerful scripting language.  The goal is to
build a platform for image processing R\&D.
\end{itemize}

\end{document}